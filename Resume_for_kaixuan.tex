\documentclass{resume} % Use the custom resume.cls style


\usepackage[left=0.75in,top=1.0in,right=0.75in,bottom=0.6in]{geometry} % Document margins
\usepackage{hyperref}
\hypersetup{
    colorlinks=true,
    %linkcolor=blue,
    %filecolor=magenta,
    urlcolor=blue,
}

\newcommand{\tab}[1]{\hspace{.2667\textwidth}\rlap{#1}}
\newcommand{\itab}[1]{\hspace{0em}\rlap{#1}}
\name{Kai Xuan}
\address{PhD student \\ Shanghai Jiao Tong University}
\address{\href{mailto:kaixuan@sjtu.edu.cn}{kaixuan@sjtu.edu.cn} \\
\href{mailto:woxuankai@gmail.com}{woxuankai@gmail.com} \\
\href{https://github.com/woxuankai}{GitHub: woxuankai}}

\begin{document}
\begin{rSection}{RESEARCH INTERESTS}
Computer Vision, Medical Image Processing, and Image Synthesis
\end{rSection}

\begin{rSection}{Education}
{\bf Shanghai Jiao Tong University (SJTU)}
\hfill {\em Sep. 2017 - Present} \\
PhD in Biomedical Engineering  \\
Advisor: Dr. Qian Wang \\
\href{http://mic.sjtu.edu.cn/}{The Medical Image Computing (MIC) Lab} \\
Institute for Medical Imaging Technology, School of Biomedical Engineering

{\bf University of Electronic Science and Technology of China (UESTC)}
\hfill {\em Sep. 2013 - Jun. 2017}
\\ BS in Biomedical Engineering \hfill {GPA:3.87/4, Ranking:4/68}
\\ UESTC Outstanding Graduate Award 
\end{rSection}

\begin{rSection}{Publications} 
\begin{enumerate}
\item Kai Xuan, Dongming Wei, Dijia Wu, Zhong Xue, Yiqiang Zhan, Weiwu Yao and Qian Wang, "Reconstruction of Isotropic High-Resolution MR image from Multiple Anisotropic Scans Using Sparse Fidelity Loss and Adversarial Regularization." In International Conference on Medical Image Computing and Computer-Assisted Intervention (MICCAI), 2019.
\end{enumerate}
\end{rSection}

\begin{rSection}{Projects}
{\bf Software toolkit for analysis of knee joint cartilage and its clinical application}\\
(Shanghai Scientific Research Plan Project)\\
Advisor: Prof. Qian Wang \\
Responsible for the segmentation of bones and cartilages from knee MR images, classification of osteoarthritis of knee from MR images, cartilage development analysis, and GUI.

{\bf Fracture detection on unfolded rib cages} \\
Mentor: Dr. Dijia Wu \\
Responsible for developing automatic rib cage unfolding algorithms
that can work on varying clinical CT data reliably and efficiently. \\
This work was done during my internship in Shanghai United~Imaging Intelligence Co., Ltd.

{\bf Segmentation of dynamic cardiac MR images using multi-atlas method} \\
(undergraduate dissertation)
\href{https://github.com/woxuankai/cardiacMRISeg/}{Source Code / Thesis} \\
Advisor: Prof. Chunming Li \\
Segmentation of left ventricular using multi-atlas method.


\end{rSection}

\end{document}
